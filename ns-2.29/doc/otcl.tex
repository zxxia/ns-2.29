\chapter{OTcl Linkage}
\label{chap:otcl:intro}

\ns\ is an object oriented simulator,
written in C++, with an OTcl interpreter as a frontend.
The simulator supports a class hierarchy in C++
(also called the compiled hierarchy in this document),
and a similar class hierarchy within the OTcl interpreter
(also called the interpreted hierarchy in this document).
The two hierarchies are closely related to each other;
from the user's perspective,
there is a one-to-one correspondence
between a class in the interpreted hierarchy
and one in the compiled hierarchy.
The root of this hierarchy is the class TclObject.
Users create new simulator objects through the interpreter;
these objects are instantiated within the interpreter, 
and are closely mirrored by a corresponding object
in the compiled hierarchy.
The interpreted class hierarchy is automatically established through
methods defined in the class TclClass.
user instantiated objects are mirrored through methods
defined in the class TclObject.
There are other hierarchies in the C++ code and OTcl scripts;
these other hierarchies are not mirrored in the manner of TclObject.

\section{Concept Overview}

\emph{Why two languages?}
\ns\ uses two languages because simulator has two different
  kinds of things it needs to do.
On one hand, detailed simulations of protocols
  requires a systems programming language
  which can efficiently manipulate bytes, packet headers,
  and implement algorithms that run over large data sets.
For these tasks run-time speed is important and
  turn-around time (run simulation, find bug, fix bug, recompile, re-run)
  is less important.

On the other hand,
  a large part of network research involves slightly varying
  parameters or configurations,
  or quickly exploring a number of scenarios.
In these cases, iteration time (change the model and re-run)
  is more important.
Since configuration runs once (at the beginning of the simulation),
  run-time of this part of the task is less important.

\ns\ meets both of these needs with two languages,
  C++ and OTcl.
C++ is fast to run but slower to change, making it suitable
  for detailed protocol implementation.
OTcl runs much slower but can be changed very quickly (and interactively),
  making it ideal for simulation configuration.
\ns\ (via \code{tclcl})
  provides glue to make objects and variables appear on both langauges.

For more information about the idea of scripting languages
  and split-language programming, see Ousterhout's article
  in IEEE Computer~\cite{Ousterhout98a}.
For more information about split level programming for network simulation,
  see the ns paper~\cite{Bajaj99a}.


\emph{Which language for what?}
Having two languages raises the question of which language should
be used for what purpose.

Our basic advice is to use OTcl:
\begin{itemize}
\item for configuration, setup, and ``one-time'' stuff
\item if you can do what you want by manipulating existing C++ objects
\end{itemize}

and use C++:
\begin{itemize}
\item if you are doing \emph{anything} that requires processing
        each packet of a flow
\item if you have to change the behavior of an existing C++ class
        in ways that weren't anticipated
\end{itemize}

For example, links are OTcl objects that assemble delay, queueing, and
possibly loss modules.  If your experiment can be done with those
pieces, great.  If instead you want do something fancier (a special
queueing dicipline or model of loss), then you'll need a new C++
object.

There are certainly grey areas in this spectrum:
most routing is done in OTcl
(although the core Dijkstra algorithm is in C++).
We've had HTTP simulations where each flow was started in OTcl
  and per-packet processing was all in C++.
This approache worked OK until we had 100s of flows
  starting per second of simulated time.
In general, if you're ever having to invoke Tcl many times per second,
  you problably should move that code to C++.


\section{Code Overview}

In this document,
we use the term ``interpreter''
to be synonymous with the OTcl interpreter.
The code to interface with the interpreter resides
in a separate directory, \code{tclcl}.
The rest of the simulator code resides in the directory, \code{ns-2}.
We will use the notation \Tclf{\tup{file}}\
to refer to a particular \tup{file}\ in the
\code{Tcl}\ directory.
Similarly, we will use the notation, \nsf{\tup{file}}
to refer to a particular \tup{file}\ in the \code{ns-2} directory.

There are a number of classes defined in \Tclf{}.
We only focus on the six that are used in \ns:
The \href{Class Tcl}{Section}{sec:Tcl} contains the methods that
C++ code will use to access the interpreter.
The \href{class TclObject}{Section}{sec:TclObject}
is the base class for all simulator objects that are also mirrored 
in the compiled hierarchy.
The \href{class TclClass}{Section}{sec:TclClass} defines
the interpreted class hierarchy, and 
the methods to permit the user to instantiate TclObjects.
The \href{class TclCommand}{Section}{sec:TclCommand}
is used to define simple global interpreter commands.
The \href{class EmbeddedTcl}{Section}{sec:EmbeddedTcl}
contains the methods to load higher level builtin commands
that make configuring simulations easier.
Finally, the \href{class InstVar}{Section}{sec:InstVar}
contains methods to access C++ member variables
as OTcl instance variables.

The procedures and functions described in this chapter can be found in
\Tclf{Tcl.\{cc, h\}}, \Tclf{Tcl2.cc}, \Tclf{tcl-object.tcl}, and,
\Tclf{tracedvar.\{cc, h\}}.
The file \Tclf{tcl2c++.c} is used in building \ns, and is mentioned
briefly in this chapter.

\section{Class Tcl}
\label{sec:Tcl}

The \clsref{Tcl}{../Tcl/Tcl.h} encapsulates the actual instance of
the OTcl interpreter, and provides the methods
to access and communicate with that interpreter.
The methods described in this section are relevant to the
\ns\ programmer who is writing C++ code.
The class provides methods for the following operations:
\begin{list}{\textbullet}{}\itemsep0pt
\item obtain a reference to the Tcl instance;
%  {\tt
%    \begin{list}{}{}
%    \item \fcn[]{Tc::instance}
%    \end{list}
%  }
\item invoke OTcl procedures through the interpreter;
%  {\tt
%    \begin{list}{}{}
%    \item \fcn[char* $s$]{Tcl::eval}
%    \item \fcn[const char* $s$]{Tcl::evalc}
%    \item \fcn[]{Tcl::eval}
%    \item \fcn[const char* $\mathit{fmt}$, \ldots]{Tcl::evalf}
%    \end{list}
%  }
\item retrieve, or pass back results to the interpreter;
%  {\tt
%    \begin{list}{}{}
%    \item \fcn[const char* $s$]{Tcl::result}
%    \item \fcn[const char* $\mathit{fmt}$, \ldots]{Tcl::resultf}
%    \item \fcn[]{Tcl::result}
%    \end{list}
%  }
\item report error situations and exit in an uniform manner; and
%  {\tt
%    \begin{list}{}{}
%    \item \fcn[const char* $s$]{Tcl::error}
%    \end{list}
%  }
\item store and lookup ``TclObjects''.
%  {\tt
%    \begin{list}{}{}
%    \item \fcn[const char* $s$]{Tcl::lookup}
%    \item \fcn[TclObject* $o$]{Tcl::enter}
%    \item \fcn[TclObject* $o$]{Tcl::remove}
%    \end{list}
%  }
\item acquire direct access to the interpreter.
%  {\tt
%    \begin{list}{}{}
%    \item \fcn[]{Tcl::interp}
%    \end{list}
%  }
\end{list}
We describe each of the methods in the following subsections.

\subsection{Obtain a Reference to the class Tcl instance}
\label{sec:instance}

A single instance of the class is declared in \Tclf{Tcl.cc}
as a static member variable;
the programmer must obtain a reference to this instance
to access other methods described in this section.
The statement required to access this instance is:
\begin{program}
        Tcl& tcl = Tcl::instance();
\end{program}

\subsection{Invoking OTcl Procedures}
\label{sec:Invoke}
There are four different methods to invoke an OTcl command
through the instance, \code{tcl}.
They differ essentially in their calling arguments.
Each function passes a string to the interpreter,
that then evaluates the string in a global context.
These methods will return to the caller if the interpreter returns TCL\_OK.
On the other hand, if the interpreter returns TCL\_ERROR,
the methods will call \proc{tkerror}.
The user can overload this procedure to selectively disregard
certain types of errors.
Such intricacies of OTcl programming are outside the
scope of this document.
\href{The next section}{Section}{sec:Result}
describes methods to access the result returned by the interpreter.
\begin{itemize}\itemsep0pt
\item \fcnref{\fcn[char* $s$]{tcl.eval}}{../Tcl/Tcl.cc}{Tcl::eval}
  invokes \fcn[]{Tcl\_GlobalEval} to execute $s$ through the interpreter.

\item \fcnref{\fcn[const char* $s$]{tcl.evalc}}{../Tcl/Tcl.cc}{Tcl::evalc}
  preserves the argument string $s$.
  It copies the string $s$ into its internal buffer;
  it then invokes the previous \fcn[char* $s$]{eval} on the internal buffer.

\item \fcnref{\fcn[]{tcl.eval}}{../Tcl/Tcl.cc}{Tcl::eval}
  assumes that the command is already stored in the class' internal
  \code{bp_}; it directly invokes \fcn[char* bp\_]{tcl.eval}.
  A handle to the buffer itself is available through the method
  \fcnref{\fcn{tcl.buffer}}{../Tcl/Tcl.h}{Tcl::buffer}.

\item
  \fcnref{\fcn[const char* $s$, \ldots]{tcl.evalf}}{../Tcl/Tcl2.cc}{Tcl::evalf}
  is a \code{Printf}(3) like equivalent.
  It uses \code{vsprintf}(3) internally to create the input string.
\end{itemize}
As an example, here are some of the ways of using the above methods:
\begin{program}
        Tcl& tcl = {\bfseries{}Tcl::instance}();
        char wrk[128];
        strcpy(wrk, "Simulator set NumberInterfaces_ 1");
        {\bfseries{}tcl.eval}(wrk);

        sprintf({\bfseries{}tcl.buffer}(), "Agent/SRM set requestFunction_ %s", "Fixed");
        {\bfseries{}tcl.eval}();

        {\bfseries{}tcl.evalc}("puts stdout {hello world}");
        {\bfseries{}tcl.evalf}("%s request %d %d", name_, sender, msgid);
\end{program}

\subsection{Passing Results to/from the Interpreter}
\label{sec:Result}

When the interpreter invokes a C++ method,
it expects the result back in the private member variable,
\code{tcl_->result}.
Two methods are available to set this variable.
\begin{list}{\textbullet}{}
\item \fcnref{\fcn[const char* $s$]{tcl.result}}{../Tcl/Tcl.h}{Tcl::result}

        Pass the result string $s$ back to the interpreter.
\item
  \fcnref{\fcn[const char* fmt, \ldots]{tcl.resultf}}{../Tcl/Tcl2.cc}{Tcl::resultf}

        \code{varargs}(3) variant of above
        to format the result using \code{vsprintf}(3),
        pass the result string back to the interpreter.
\end{list}
\begin{program}
        if (strcmp(argv[1], "now") == 0) \{
                {\bfseries{}tcl.resultf}("%.17g", clock());
                return TCL_OK;
        \}
        {\bfseries{}tcl.result}("Invalid operation specified");
        return TCL_ERROR;
\end{program}

Likewise, when a C++ method invokes an OTcl command,
the interpreter returns the result in \code{tcl_->result}.
\begin{list}{\textbullet}{}
\item \fcnref{\fcn{tcl.result}}{../Tcl/Tcl.h}{Tcl::result}
      must be used to retrieve the result.
      Note that the result is a string, that must be converted
      into an internal format appropriate to the type of result.
\end{list}
\begin{program}
        tcl.evalc("Simulator set NumberInterfaces_");
        char* ni = {\bfseries{}tcl.result}();
        if (atoi(ni) != 1)
                tcl.evalc("Simulator set NumberInterfaces_ 1");
\end{program}
        
\subsection{Error Reporting and Exit}
\label{sec:ErrorReporting}

This method provides a uniform way to report errors in the compiled code.
\begin{list}{\textbullet}{}
\item \fcnref{\fcn[const char* $s$]{tcl.error}}{../Tcl/Tcl.cc}{Tcl::error}
performs the following functions:
write $s$ to stdout; write \code{tcl_->result} to stdout;
exit with error code 1.
\end{list}
\begin{program}
        {\bfseries{}tcl.resultf}("cmd = %s", cmd);
        {\bfseries{}tcl.error}("invalid command specified");
        /*{\cf{}NOTREACHED}*/
\end{program}

Note that
there are minor differences between returning TCL\_ERROR
\href{as we did in the previous subsection}{Section}{sec:Result},
and calling \fcn[]{Tcl::error}.
The former generates an exception within the interpreter;
the user can trap the exception and possibly recover from the error.
If the user has not specified any traps, 
the interpreter will print a stack trace and exit.
However, if the code invokes \fcn[]{error},
then the simulation user cannot trap the error;
in addition, \ns\ will not print any stack trace.

\subsection{Hash Functions within the Interpreter}
\label{sec:HashTables}

\ns\ stores a reference to every TclObject in the compiled hierarchy
in a hash table;
this permits quick access to the objects.
The hash table is internal to the interpreter.
\ns\ uses the name of the \code{TclObject} as the key
to enter, lookup, or delete the TclObject in the hash table.
\begin{list}{\textbullet}{}
\item \fcnref{\fcn[TclObject* $o$]{tcl.enter}}{../Tcl/Tcl.cc}{Tcl::enter}
  will insert a pointer to the TclObject $o$ into the hash table.

  It is used by
  \fcnref{\fcn[]{TclClass::create\_shadow}}{../Tcl/Tcl.cc}{TclClass::create\_shadow}
  to insert an object into the table, when that object is created.

\item \fcnref{\fcn[char* $s$]{tcl.lookup}}{../Tcl/Tcl.h}{Tcl::lookup}
  will retrieve the TclObject with the name $s$.

  It is used by
  \fcnref{\fcn[]{TclObject::lookup}}{../Tcl/Tcl.h}{TclObject::lookup}.
\item \fcnref{\fcn[TclObject* $o$]{tcl.remove}}{../Tcl/Tcl.cc}{Tcl::remove}
  will delete references to the TclObject $o$ from the hash table.

  It is used by
  \fcnref{\fcn[]{TclClass::delete\_shadow}}{../Tcl/Tcl.cc}{TclClass::delete\_shadow}
  to remove an existing entry from the hash table,
  when that object is deleted.
\end{list}
These functions are used internally by
the class TclObject and class TclClass.

\subsection{Other Operations on the Interpreter}
\label{sec:otcl:other}

If the above methods are not sufficient,
then we must acquire the handle to the interpreter,
and write our own functions.
\begin{list}{\textbullet}{}
\item \fcnref{\fcn{tcl.interp}}{../Tcl/Tcl.h}{Tcl::interp}
        returns the handle to the interpreter that is stored
        within the class Tcl.
\end{list}

\section{Class TclObject}
\label{sec:TclObject}

\clsref{TclObject}{../Tcl/Tcl.h}
is the base class for most of the other classes
in the interpreted and compiled hierarchies.
Every object in the class TclObject is created by the user
from within the interpreter.
An equivalent shadow object is created in the compiled hierarchy.
The two objects are closely associated with each other.
The class TclClass, described in the next section,
contains the mechanisms that perform this shadowing.

In the rest of this document, we often refer to an object as a TclObject%
\footnote{In the latest release of \ns\ and \nsTcl,
  this object has been renamed to \code{SplitObjefct},
  which more accurately reflects its nature of existence.
  However, for the moment,
  we will continue to use the term TclObject
  to refer to these objects and this class.}.
By this, we refer to a particular object that is either in the class
TclObject, or in a class that is derived from the class TclObject.
If it is necessary, we will explicitly qualify whether that object is
an object within the interpreter, or an object within the compiled code.
In such cases,
we will use the abbreviations ``interpreted object'', and
``compiled object'' to distinguish the two.
and within the compiled code respectively.

\paragraph{Differences from \ns~v1}
Unlike \ns~v1, the class TclObject
subsumes the earlier functions of the NsObject class.
It therefore stores the
\href{interface variable bindings}{Section}{sec:VarBinds}
that tie OTcl instance variables in the interpreted object
to corresponding C++ member variables in the compiled object.
The binding is stronger than in \ns~v1 in that
any changes to the OTcl variables are trapped,
and the current C++ and OTcl values
are made consistent after each access through the interpreter.
The consistency is done through the
\href{class InstVar}{Section}{sec:InstVar}.
Also unlike \ns~v1, objects in the class TclObject
are no longer stored as a global link list.
Instead, they are stored in a hash table in the
\href{class Tcl}{Section}{sec:HashTables}.

\paragraph{Example configuration of a TclObject}
The following example illustrates the configuration of
an SRM agent (\clsref{Agent/SRM/Adaptive}{../ns-2/srm-adaptive.tcl}).
\begin{program}
        set srm [new Agent/SRM/Adaptive]
        \$srm set packetSize_ 1024
        \$srm traffic-source \$s0
\end{program}
By convention in \ns,
the class Agent/SRM/Adaptive is a subclass of Agent/SRM,
is a subclass of Agent, is a subclass of TclObject.
The corresponding compiled class hierarchy is
the ASRMAgent, derived from SRMAgent, derived from Agent,
derived from TclObject respectively.
The first line of the above example shows how a TclObject is 
\href{created (or destroyed)}{Section}{sec:Creation};
the next line configures
\href{a bound variable}{Section}{sec:VarBinds};
and finally, the last line illustrates
the interpreted object invoking a C++ method
\href{as if they were an instance procedure}{Section}{sec:Commands}.

\subsection{Creating and Destroying TclObjects}
\label{sec:Creation}

When the user creates a new TclObject,
using the procedures \proc[]{new} and \proc[]{delete};
these procedures are defined in \Tclf{tcl-object.tcl}.
They can be used to create and destroy objects in all classes,
including TclObjects.%
\footnote{As an example, the classes Simulator, Node, Link, or rtObject,
are classes that are \emph{not} derived from the class TclObject.
Objects in these classes  are not, therefore, TclObjects.
However, a Simulator, Node, Link, or route Object is also instantiated
using the \code{new} procedure in \ns.}.
In this section,
we describe the internal actions executed when a TclObject
is created.

\paragraph{Creating TclObjects}
By using \proc[]{new}, the user creates an interpreted TclObject.
the interpreter will execute the constructor for that object, \proc[]{init},
passing it any arguments provided by the user.
\ns\ is responsible for automatically  creating the compiled object.
The shadow object gets created by the base class TclObject's constructor.
Therefore, the constructor for the new TclObject
must call the parent class constructor first.
\proc[]{new} returns a handle to the object, that can then be used
for further operations upon that object.

The following example illustrates the Agent/SRM/Adaptive constructor:
\begin{program}
        Agent/SRM/Adaptive instproc init args \{
                eval \$self next \$args
                \$self array set closest_ "requestor 0 repairor 0"
                \$self set eps_    [\$class set eps_]
        \}
\end{program}

The following sequence of actions are performed by the interpreter
as part of instantiating a new TclObject.
For ease of exposition, we describe the steps that are executed
to create an Agent/SRM/Adaptive object.
The steps are:
\begin{enumerate}
\item
  Obtain an unique handle for the new object   from the TclObject name space.
  The handle is returned to the user.
  Most handles in \ns\ have the form \code{_o\tup{NNN}}, where \tup{NNN}
  is an integer.  This handle is created by
  \fcnref{\proc{getid}}{../tclcl/tcl-object.tcl}{TclObject::getid}.
  It can be retrieved from C++ with the
  \fcnref{\proc{name()}}{../tclcl/tclcl.h}{TclObject::name()}
  method.
\item Execute the constructor for the new object.
  Any user-specified arguments are passed as arguments to the constructor.
  This constructor must invoke the constructor
  associated with its parent class.

  In our example above, the Agent/SRM/Adaptive calls its parent class
  in the very first line.  

  Note that each constructor,
  in turn invokes its parent class' constructor \textit{ad nauseum}.
  The last constructor in \ns\ is
  \fcnref{the TclObject constructor}{../Tcl/tcl-object.tcl}{TclObject::init}.
  This constructor is responsible for setting up the shadow object, and 
  performing other initializations and bindings, as we explain below.
  \emph{It is preferable to call the parent constructors first before
    performing the initializations required in this class.}
  This allows the shadow objects to be set up,
  and the variable bindings established.
\item The TclObject constructor invokes the instance procedure
  \proc[]{create-shadow} for the class Agent/SRM/Adaptive.
\item When the shadow object is created,
  \ns\ calls all of the constructors for the compiled object,
  each of which may establish variable bindings for objects in that class,
  and perform other necessary initializations.
  Hence our earlier injunction that it is preferable to invoke the parent
  constructors prior to performing the class initializations.
\item After the shadow object is successfully created,
  \fcnref{\fcn{create\_shadow}}{../Tcl/Tcl.cc}{TclClass::create\_shadow}
  \begin{enumerate}
  \item adds the new object to hash table of TclObjects
        \href{described earlier}{Section}{sec:HashTables}.
  \item makes \proc[]{cmd} an instance procedure of the newly created
    interpreted object.
    This instance procedure
    invokes the \fcn[]{command} method of the compiled object.
    In \href{a later subsection}{Section}{sec:Commands},
    we describe how the \code{command} method is defined, and invoked.
  \end{enumerate}
\end{enumerate}
Note that all of the above shadowing mechanisms only work when
the user creates a new TclObject through the interpreter.
It will not work if the programmer creates a compiled TclObject unilaterally.
Therefore, the programmer is enjoined not to use the C++ new method
to create compiled objects directly.

\paragraph{Deletion of TclObjects}
The \code{delete} operation
destroys the interpreted object, and the corresponding shadow object.
For example,
\fcnref{\proc[\tup{scheduler}]{use-scheduler}}{%
  ../ns-2/ns-lib.tcl}{Simulator::use-scheduler}
uses the \code{delete} procedure to remove the default list scheduler,
and instantiate an alternate scheduler in its place.
\begin{program}
        Simulator instproc use-scheduler type \{
                $self instvar scheduler_

                delete scheduler_ \; first delete the existing list scheduler;
                set scheduler_ [new Scheduler/$type]
        \}
\end{program}

As with the constructor, the object destructor must call the destructor
for the parent class explicitly as the very last statement of the destructor.
The TclObject destructor
will invoke the instance procedure \code{delete-shadow},
that in turn invokes \fcnref{the equivalent compiled method}{%
  ../Tcl/Tcl.cc}{TclClass::delete\_shadow}
to destroy the shadow object.
The interpreter itself will destroy the interpreted object.

\subsection{Variable Bindings}
\label{sec:VarBinds}

In most cases,
access to compiled member variables is restricted to compiled code,
and access to interpreted member variables is likewise
confined to access via interpreted code;
however, it is possible to establish bi-directional bindings
such that both the interpreted member variable
and the compiled member variable access the same data, 
and changing the value of either variable
changes the value of the corresponding paired variable to same value.

The binding is established by the compiled constructor
when that object is instantiated;
it is automatically accessible by the interpreted object as 
an instance variable.
\ns\ supports five different data types: reals, bandwidth valued variables, 
time valued variables, integers, and booleans.
The syntax of how these values can be specified in OTcl is different
for each variable type.
\begin{itemize}\itemsep0pt
\item Real and Integer valued variables are specified in the ``normal'' form.
        For example,
        \begin{program}
        $object set realvar 1.2e3
        $object set intvar  12
        \end{program}
\item Bandwidth is specified as a real value, optionally
  suffixed by a `k' or `K' to mean kilo-quantities, or `m' or `M' to
  mean mega-quantities.
  A final optional suffix of `B' indicates that the quantity expressed
  is in Bytes per second.
  The default is bandwidth expressed in bits per second.
        For example, all of the following are equivalent:
        \begin{program}
        $object set bwvar 1.5m
        $object set bwvar 1.5mb
        $object set bwvar 1500k
        $object set bwvar 1500kb
        $object set bwvar .1875MB
        $object set bwvar 187.5kB
        \$object set bwvar 1.5e6
        \end{program}

\item Time is specified as a real value, optionally suffixed by a
  `m' to express time in milli-seconds, `n' to express time in
  nano-seconds, or `p' to express time in pico-seconds.
  The default is time expressed in seconds.
        For example, all of the following are equivalent:
        \begin{program}
        $object set timevar 1500m
        $object set timevar 1.5
        $object set timevar 1.5e9n
        $object set timevar 1500e9p
        \end{program}
  Note that we can also safely add a $s$ to reflect the time unit of seconds.
  \ns\ will ignore anything other than a valid real number specification,
  or a trailing `m', `n', or `p'.

\item Booleans can be expressed either as an integer, or as `T' or `t'
  for true.  Subsequent characters after the first letter are ignored.
  If the value is neither an integer, nor a true value,
  then it is assumed to be false.
        For example,
        \begin{program}
        \$object set boolvar t           \; set to true;
        \$object set boolvar true
        \$object set boolvar 1   \; or any non-zero value;

        \$object set boolvar false       \; set to false;
        \$object set boolvar junk        
        \$object set boolvar 0
        \end{program}

\end{itemize}

The following example shows the constructor for the ASRMAgent%
\footnote{Note that this constructor is embellished to illustrate
        the features of the variable binding mechanism.}.
\begin{program}
        ASRMAgent::ASRMAgent() \{
                bind("pdistance_", &pdistance_);      \* real variable */
                bind("requestor_", &requestor_);      \* integer variable */
                bind_time("lastSent_", &lastSessSent_); \* time variable */
                bind_bw("ctrlLimit_", &ctrlBWLimit_); \* bandwidth variable */
                bind_bool("running_", &running_);     \* boolean variable */
        \}
\end{program}
Note that all of the functions above take two arguments,
the name of an OTcl variable,
and the address of the corresponding compiled member variable
that is linked.
While it is often the case that these bindings are established
by the constructor of the object, 
it need not always be done in this manner.
We will discuss such alternate methods
when we describe \href{the class InstVar}{Section}{sec:InstVar}
in detail later.

Each of the variables that is bound is automatically initialised
with default values when the object is created.
The default values are specified as interpreted class variables.
This initialisation is done by the routing \proc[]{init-instvar},
invoked by methods in the class Instvar,
\href{described later}{Section}{sec:InstVar}.
\proc[]{init-instvar} checks the class of the interpreted object,
and all of the parent class of that object, to find the first
class in which the variable is defined.
It uses the value of the variable in that class to initialise the object.
Most of the bind initialisation values are defined in
\nsf{tcl/lib/ns-default.tcl}.

For example, if the following class variables are defined for the ASRMAgent:
\begin{program}
        Agent/SRM/Adaptive set pdistance_ 15.0
        Agent/SRM set pdistance_ 10.0
        Agent/SRM set lastSent_ 8.345m
        Agent set ctrlLimit_    1.44M
        Agent/SRM/Adaptive set running_ f
\end{program}
Therefore, every new Agent/SRM/Adaptive object will have
\code{pdistance_} set to 15.0;
\code{lastSent_} is set to 8.345m
from the setting of the class variable of the parent class;
\code{ctrlLimit_} is set to 1.44M
using the class variable of the parent class twice removed;
\code{running} is set to false;
the instance variable \code{pdistance_} is not initialised,
because no class variable
exists in any of the class hierarchy of the interpreted object.
In such instance, \proc[]{init-instvar} will invoke 
\proc[]{warn-instvar}, to print out a warning about such a variable.
The user can selectively override this procedure
in their simulation scripts, to elide this warning.

Note that the actual binding
is done by instantiating objects in the class InstVar.
Each object in the class InstVar binds 
one compiled member variable to one interpreted member variable.
A TclObject stores a list of InstVar objects corresponding
to each of its member variable that is bound in this fashion.
The head of this list is stored in its member variable
\code{instvar_} of the TclObject.

One last point to consider is that
\ns\ will guarantee that the actual values
of the variable, both in the interpreted object and the compiled object,
will be identical at all times.
However, if there are methods and other variables
of the compiled object that track the value of this variable,
they must be explicitly invoked or changed whenever the
value of this variable is changed.
This usually requires additional primitives that the user should invoke.
One way of providing such primitives in \ns\ is through
the \fcn[]{command} method described in the next section.


\subsection{Variable Tracing}
\label{sec:VarTrace}

In addition to variable bindings, TclObject also supports tracing of
both C++ and Tcl instance variables.  A traced variable can be created
and configured either in C++ or Tcl.  To establish variable tracing at
the Tcl level, the variable must be visible in Tcl, which means that it
must be a bounded C++/Tcl or a pure Tcl instance variable.  In addition,
the object that owns the traced variable is also required to establish
tracing using the Tcl \code{trace} method of TclObject.  The first
argument to the \code{trace} method must be the name of the variable.
The optional second argument specifies the trace object that is
responsible for tracing that variable.  If the trace object is not
specified, the object that own the variable is responsible for tracing
it.

For a TclObject to trace variables, it must extend the C++
\code{trace} method that is virtually defined in TclObject.  The Trace
class implements a simple \code{trace} method, thereby, it can act as a
generic tracer for variables.

\begin{verbatim}
class Trace : public Connector {
        ...
        virtual void trace(TracedVar*);
};
\end{verbatim}

Below is a simple example for setting up variable tracing in Tcl:

\begin{small}
\begin{verbatim}
        # \$tcp tracing its own variable cwnd_
        \$tcp trace cwnd_

        # the variable ssthresh_ of \$tcp is traced by a generic \$tracer
        set tracer [new Trace/Var]
        \$tcp trace ssthresh_ \$tracer
\end{verbatim}
\end{small}

For a C++ variable to be traceable, it must belong to a class that
derives from TracedVar.  The virtual base class TracedVar keeps track of
the variable's name, owner, and tracer.  Classes that derives from
TracedVar must implement the virtual method \code{value}, that takes a
character buffer as an argument and writes the value of the variable
into that buffer.

\begin{small}
\begin{verbatim}
class TracedVar {
        ...
        virtual char* value(char* buf) = 0;
protected:
        TracedVar(const char* name);
        const char* name_;      // name of the variable
        TclObject* owner_;      // the object that owns this variable
        TclObject* tracer_;     // callback when the variable is changed
        ...
};
\end{verbatim}
\end{small}

The TclCL library exports two classes of TracedVar:  \code{TracedInt} and
\code{TracedDouble}.  These classes can be used in place of the basic
type int and double respectively.  Both TracedInt and TracedDouble
overload all the operators that can change the value of the variable
such as assignment, increment, and decrement.  These overloaded
operators use the \code{assign} method to assign the new value to the
variable and call the tracer if the new value is different from the old
one.  TracedInt and TracedDouble also implement their \code{value}
methods that output the value of the variable into string.  The width
and precision of the output can be pre-specified.

\subsection{\code{command} Methods: Definition and Invocation}
\label{sec:Commands}

For every TclObject that is created, \ns\ establishes
the instance procedure, \proc[]{cmd},
as a hook to executing methods through the compiled shadow object.
The procedure \proc[]{cmd} invokes the method \fcn[]{command}
of the shadow object automatically, passing the arguments to \proc[]{cmd}
as an argument vector to the \fcn[]{command} method.

The user can invoke the \proc[]{cmd} method in one of two ways:
by explicitly invoking the procedure, specifying the desired
operation as the first argument, or
implicitly, as if there were an instance procedure of the same name as the
desired operation.
Most simulation scripts will use the latter form, hence, we will
describe that mode of invocation first.

Consider the that the distance computation in SRM is done by
the compiled object; however, it is often used by the interpreted object.
It is usually invoked as:
\begin{program}
        \$srmObject distance? \tup{agentAddress}
\end{program}
If there is no instance procedure called \code{distance?},
the interpreter will invoke the instance procedure
\proc[]{unknown}, defined in the base class TclObject.
The unknown procedure then invokes
\begin{program}
        \$srmObject cmd distance? \tup{agentAddress}
\end{program}
to execute the operation through the compiled object's
\fcn[]{command} procedure.

Ofcourse, the user could explicitly invoke the operation directly.
One reason for this might be to overload the operation by using
an instance procedure of the same name.
For example,
\begin{program}
        Agent/SRM/Adaptive instproc distance? addr \{
                \$self instvar distanceCache_
                if ![info exists distanceCache_(\$addr)] \{
                        set distanceCache_(\$addr) [{\bfseries{}\$self cmd distance? \$addr}]
                \}
                set distanceCache_(\$addr)
        \}
\end{program}

We now illustrate how the \fcn[]{command} method using
\fcn[]{ASRMAgent::command} as an example.
\begin{program}
        int ASRMAgent::command(int argc, const char*const*argv) \{
                Tcl& tcl = Tcl::instance();
                if (argc == 3) \{
                        if (strcmp(argv[1], "distance?") == 0) \{
                                int sender = atoi(argv[2]);
                                SRMinfo* sp = get_state(sender);
                                tcl.tesultf("%f", sp->distance_);
                                return TCL_OK;
                        \}
                \}
                return (SRMAgent::command(argc, argv));
        \}
\end{program}
We can make the following observations from this piece of code:
\begin{itemize}
\item The function is called with two arguments:
  
  The first argument (\code{argc}) indicates
  the number of arguments specified in the command line to the interpreter.

  The command line arguments vector (\code{argv}) consists of
  
  --- \code{argv[0]} contains the name of the method, ``\code{cmd}''.

  --- \code{argv[1]} specifies the desired operation.

  --- If the user specified any arguments, then they are placed in
  \code{argv[2\ldots(argc - 1)]}.

  The arguments are passed as strings;
  they must be converted to the appropriate data type.
\item If the operation is successfully matched,
  the match should return the result of the operation
  using methods \href{described earlier}{Section}{sec:Result}.
\item \fcn[]{command} itself must return either \code{TCL_OK} or \code{TCL_ERROR}
  to indicate success or failure as its return code.
\item If the operation is not matched in this method, it must
  invoke its parent's command method, and return the corresponding result.

  This permits the user to concieve of operations as having the same
  inheritance properties as instance procedures or compiled methods.

  In the event that this \code{command} method 
  is defined for a class with multiple inheritance,
  the programmer has the liberty to choose one of two implementations:

  1) Either they can invoke one of the parent's \code{command} method,
  and return the result of that invocation, or

  2) They can each of the parent's \code{command} methods in some sequence,
  and return the result of the first invocation that is successful.
  If none of them are successful, then they should return an error.
\end{itemize}
In our document, we call operations executed through the 
\fcn[]{command} \emph{instproc-like}s.
This reflects the usage of these operations as if they were
OTcl instance procedures of an object,
but can be very subtly different in their realisation and usage.


\section{Class TclClass}
\label{sec:TclClass}

This compiled class (\clsref{TclClass}{../Tcl/Tcl.h})
is a pure virtual class.
Classes derived from this base class provide two functions:
construct the interpreted class hierarchy
to mirror the compiled class hierarchy; and
provide methods to instantiate new TclObjects.
Each such derived class is associated with a particular compiled class
in the compiled class hierarchy, and can instantiate new objects in the
associated class.

As an example, consider a class such as the
class \code{RenoTcpClass}.
It is derived from class \code{TclClass}, and
is associated with the class \code{RenoTcpAgent}.
It will instantiate new objects in the class \code{RenoTcpAgent}.
The compiled class hierarchy for \code{RenoTcpAgent} is that
it derives from \code{TcpAgent}, that in turn derives from \code{Agent},
that in turn derives (roughly) from \code{TclObject}.
\code{RenoTcpClass} is defined as
\begin{program}
        static class RenoTcpClass: public TclClass \{
        public:
                RenoTcpClass() : TclClass("Agent/TCP/Reno") \{\}
                TclObject* create(int argc, const char*const* argv) \{
                        return (new RenoTcpAgent());
                \}
        \} class_reno;
\end{program}
We can make the following observations from this definition:
\begin{enumerate}
\item The class defines only the constructor, and one additional method,
  to \code{create} instances of the associated TclObject.
\item \ns\ will execute the \code{RenoTcpClass} constructor
  for the static variable \code{class_reno}, when it is first started.
  This sets up the appropriate methods and the interpreted class hierarchy.
\item The constructor specifies the interpreted class explicitly as
  \code{Agent/TCP/Reno}.  This also specifies the interpreted class
  hierarchy implicitly.

  Recall that the convention in \ns\ is to use
  the character slash ('/') is a separator.
  For any given class \code{A/B/C/D},
  the class \code{A/B/C/D} is a sub-class of \code{A/B/C},
  that is itself a sub-class of \code{A/B},
  that, in turn, is a sub-class of \code{A}.
  \code{A} itself is a sub-class of \code{TclObject}.

  In our case above, the TclClass constructor creates three classes,
  \code{Agent/TCP/Reno} sub-class of \code{Agent/TCP}
  sub-class of \code{Agent} sub-class of \code{TclObject}.
\item This class is associated with the class \code{RenoTcpAgent};
  it creats new objects in this associated class.
\item The \code{RenoTcpClass::create} method returns TclObjects in the
  class \code{RenoTcpAgent}.
\item When the user specifies \code{new Agent/TCP/Reno},
  the routine \code{RenoTcpClass::create} is invoked.
\item The arguments vector (\code{argv}) consists of

  --- \code{argv[0]} contains the name of the object.

  --- \code{argv[1\ldots3]} contain
  \code{$self}, \code{$class}, and \code{$proc}.%$
  Since \code{create} is called
  through the instance procedure \code{create-shadow},
  \code{argv[3]} contains \code{create-shadow}.

  --- \code{argv[4]}
  contain any additional arguments (passed as a string) provided by the user.
\end{enumerate}
The \clsref{Trace}{../ns-2/trace.cc} illustrates
argument handling by TclClass methods.
\begin{program}
        class TraceClass : public TclClass \{
        public:
                TraceClass() : TclClass("Trace") \{\}
                TclObject* create(int args, const char*const* argv) \{
                        if (args >= 5)
                                return (new Trace(*argv[4]));
                        else
                                return NULL;
                \}
        \} trace_class;
\end{program}
A new Trace object is created as
\begin{program}
        new Trace "X"
\end{program}
Finally, the nitty-gritty details of how the 
interpreted class hierarchy is constructed:
\begin{enumerate}
\item The object constructor is executed when \ns\ first starts.
\item This constructor calls the TclClass constructor
  with the name of the interpreted class as its argument.
\item The TclClass constructor stores the name of the class,
  and inserts this object into a linked list of the TclClass objects.
\item During initialization of the simulator,
  \fcnref{\fcn{Tcl\_AppInit}}{../ns-2/ns_tclsh.cc}{::Tcl\_AppInit}
  invokes 
  \fcnref{\fcn{TclClass::bind}}{../Tcl/Tcl.cc}{TclClass::bind}
\item For each object in the list of TclClass objects,
  \fcn[]{bind} invokes 
  \fcnref{\proc[]{register}}{../Tcl/tcl-object.tcl}{TclObject::register},
  specifying the name of the interpreted class as its argument.
\item \proc[]{register} establishes the class hierarchy,
  creating the classes that are required, and not yet created.
\item Finally, \fcn[]{bind} defines instance procedures
  \code{create-shadow} and \code{delete-shadow} for this new class.
\end{enumerate}

\subsection{How to Bind Static C++ Class Member Variables}

In Section~\ref{sec:TclObject}, we have seen how to expose member
variables of a C++ object into OTcl space. 
This, however, does not apply to static member variables of a C++
class. 
Of course, one may create an OTcl variable for the static member
variable of every C++ object; obviously this defeats the whole meaning
of static members.

We cannot solve this binding problem using a similar solution as
binding in TclObject, which is based on InstVar, because InstVars in
TclCL require the presence of a TclObject.
However, we can create a method of the corresponding TclClass and
access static members of a C++ class through the methods of its
corresponding TclClass.
The procedure is as follows:
\begin{enumerate}
\item Create your own derived TclClass as described above;
\item Declare methods \fcn[]{bind} and \fcn[]{method} in your derived
  class;
\item Create your binding methods in the implementation of your
  \fcn[]{bind} with \code{add_method("your_method")}, then implement
  the handler in \fcn[]{method} in a similar way as you would do in
  \fcn[]{TclObject::command}. 
  Notice that the number of arguments passed to
  \fcn[]{TclClass::method} are different from those passed to
  \fcn[]{TclObject::command}.
  The former has two more arguments in the front. 
\end{enumerate}

As an example, we show a simplified version of
\code{PacketHeaderClass} in \nsf{packet.cc}. 
Suppose we have the following class \code{Packet} which has a static
variable \code{hdrlen_} that we want to access from OTcl:
\begin{program}
class Packet \{
        ......
        static int hdrlen_;
\};
\end{program}
Then we do the following to construct an accessor for this variable:
\begin{program}
class PacketHeaderClass : public TclClass \{
protected:
        PacketHeaderClass(const char* classname, int hdrsize);
        TclObject* create(int argc, const char*const* argv);
        \* These two implements OTcl class access methods */
        virtual void bind();
        virtual int method(int argc, const char*const* argv);
\};

void PacketHeaderClass::bind()
\{
        \* Call to base class bind() must precede add_method() */
        TclClass::bind();
        add_method("hdrlen");
\}

int PacketHeaderClass::method(int ac, const char*const* av)
\{
        Tcl& tcl = Tcl::instance();
        \* Notice this argument translation; we can then handle them \
as if in TclObject::command() */
        int argc = ac - 2;
        const char*const* argv = av + 2;
        if (argc == 2) \{
                if (strcmp(argv[1], "hdrlen") == 0) \{
                        tcl.resultf("%d", Packet::hdrlen_);
                        return (TCL_OK);
                \}
        \} else if (argc == 3) \{
                if (strcmp(argv[1], "hdrlen") == 0) \{
                        Packet::hdrlen_ = atoi(argv[2]);
                        return (TCL_OK);
                \}
        \}
        return TclClass::method(ac, av);
\}
\end{program}
After this, we can then use the following OTcl command to access and
change values of \code{Packet::hdrlen_}:
\begin{program}
        PacketHeader hdrlen 120
        set i [PacketHeader hdrlen]
\end{program}

\section{Class TclCommand}
\label{sec:TclCommand}

This class (\clsref{TclCommand}{../Tcl/Tcl.h})
provides just the mechanism for \ns\ to export
simple commands to the interpreter, 
that can then be executed within a global context by the interpreter.
There are two functions defined in \nsf{misc.cc}:
\code{ns-random} and \code{ns-version}.
These two functions are initialized by the function
\fcnref{\fcn{init\_misc}}{../ns-2/misc.cc}{::init\_misc},
defined in \nsf{misc.cc};
\code{init_misc} is invoked by
\fcnref{\fcn{Tcl\_AppInit}}{../ns-2/ns_tclsh.cc}{::Tcl\_AppInit}
during startup.
\begin{itemize}\itemsep0pt
\item \clsref{VersionCommand}{../ns-2/misc.cc}
  defines the command \code{ns-version}.
  It takes no argument, and returns the current \ns\ version string.
\begin{program}
            % ns-version                \; get the current version;
            2.0a12
\end{program}

\item \clsref{RandomCommand}{../ns-2/misc.cc}
  defines the command \code{ns-random}.
  With no argument, \code{ns-random} returns an integer,
  uniformly distributed in the interval $[0, 2^{31}-1]$.

  When specified an argument, it takes that argument as the seed.
  If this seed value is 0, the command uses a heuristic seed value;
  otherwise, it sets the seed for the random number generator to the
  specified value.
\begin{program}
            % ns-random                 \; return a random number;
            2078917053
            % ns-random 0               \;set the seed heuristically;
            858190129
            % ns-random 23786           \;set seed to specified value;
            23786
\end{program}
\end{itemize}

\emph{Note that, it is generally not advisable to construct
  top-level commands that are available to the user.}
We now describe how to define a new command
using the example \code{class say_hello}.
The example defines the command \code{hi},
to print the string ``hello world'',
followed by any command line arguments specified by the user.
For example,
\begin{program}
            % hi this is ns [ns-version]
            hello world, this is ns 2.0a12
\end{program}
\begin{enumerate}
\item The command must be defined within a class
  derived from the \clsref{TclCommand}{../Tcl/Tcl.h}.
  The class definition is:
  \begin{program}
        class say_hello : public TclCommand \{
        public:
                say_hello();
                int command(int argc, const char*const* argv);
        \};
  \end{program}
\item The constructor for the class must invoke the
  \fcnref{TclCommand constructor}{../Tcl/Tcl.cc}{TclCommand::TclCommand}
  with the command as argument; \ie,
  \begin{program}
        say_hello() : TclCommand("hi") \{\}
  \end{program}
  The \code{TclCommand} constructor sets up "hi"
  as a global procedure that invokes
  \fcnref{\fcn[]{TclCommand::dispatch\_cmd}}{../ns-2/Tcl.cc}{TclCommand::dispatch\_cmd}.
\item  The method \fcn[]{command} must perform the desired action.

  The method is passed two arguments.  The first argument, \code{argc},
  contains the number of actual arguments passed by the user.

  The actual arguments passed by the user are passed as an
  argument vector (\code{argv}) and contains the following:
  
  --- \code{argv[0]} contains the name of the command (\code{hi}).

  --- \code{argv[1\ldots(argc - 1)]} contains additional arguments
  specified on the command line by the user.

  \fcn[]{command} is invoked by \fcn[]{dispatch\_cmd}.
\begin{program}
        #include <streams.h>        \* because we are using stream I/O */
        
        int say_hello::command(int argc, const char*const* argv) \{
                cout << "hello world:";
                for (int i = 1; i < argc; i++)
                        cout << ' ' << argv[i];
                cout << '\bs n';
                return TCL_OK;
        \}
\end{program}
\item Finally, we require an instance of this class.
  \code{TclCommand} instances are created in the routine
  \fcnref{\fcn{init\_misc}}{../ns-2/misc.cc}{::init\_misc}.
  \begin{program}
        new say_hello;
  \end{program}
\end{enumerate}
Note that there used to be more functions such as \code{ns-at}\ and
\code{ns-now}\ that were accessible in this manner.
Most of these functions have been subsumed into existing classes.
In particular, \code{ns-at}\ and \code{ns-now}\ are accessible
through the
\fcnref{scheduler TclObject}{../ns-2/scheduler.cc}{Scheduler::command}.
These functions are defined in \nsf{tcl/lib/ns-lib.tcl}.
\begin{program}
            % set ns [new Simulator]    \; get new instance of simulator;
            _o1
            % $ns now                   \; query simulator for current time;
            0
            % $ns at \ldots             \; specify at operations for simulator;
            \ldots
\end{program}
          

\section{Class EmbeddedTcl}
\label{sec:EmbeddedTcl}

\ns\ permits the development of functionality in either compiled code,
or through interpreter code, that is evaluated at initialization.
For example, the scripts \Tclf{tcl-object.tcl} or the scripts in
\nsf{tcl/lib}.
Such loading and evaluation of scripts is done through objects in the
\clsref{EmbeddedTcl}{../Tcl/Tcl.h}.

The easiest way to extend \ns\ is to add OTcl code
to either \Tclf{tcl-object.tcl} or through scripts
in the \nsf{tcl/lib} directory.
Note that, in the latter case, \ns\ sources
\nsf{tcl/lib/ns-lib.tcl} automatically, and hence
the programmer must add a couple of lines to this file
so that their script will also get automatically sourced by \ns\
at startup.
As an example,
the file \nsf{tcl/mcast/srm.tcl} defines some of the instance procedures
to run SRM.
In \nsf{tcl/lib/ns-lib.tcl}, we have the lines:
\begin{program}
        source tcl/mcast/srm.tcl
\end{program}
to automatically get srm.tcl sourced by \ns\ at startup.

Three points to note with EmbeddedTcl code are that
firstly, if the code has an error that is caught during the eval,
then \ns\ will not run.
Secondly, the user can explicitly override any of the code in the scripts.
In particular, they can re-source the entire script after making their own
changes. 
Finally, after adding the scripts to \nsf{tcl/lib/ns-lib.tcl}, and
every time thereafter that they change their script, the user
must recompile \ns\ for their changes to take effect.
Of course, in most cases\footnote{%
The few places where this might not work
are when certain variables might have to be defined or undefined,
or otherwise the script contains code
other than procedure and variable definitions and 
executes actions directly that might not be reversible.},
the user can source their script
to override the embedded code.


The rest of this subsection illustrate
how to integrate individual scripts directly into \ns.
The first step is convert the script into an EmbeddedTcl object.
The lines below expand ns-lib.tcl and create the EmbeddedTcl object
instance called \code{et_ns_lib}:
\begin{program}
        tclsh bin/tcl-expand.tcl tcl/lib/ns-lib.tcl | \bs
                               ../Tcl/tcl2c++ et_ns_lib > gen/ns_tcl.cc
\end{program}
The script, \xref{\nsf{bin/tcl-expand.tcl}}{../ns-2/tcl-expand.tcl}
expands \code{ns-lib.tcl} by replacing all \code{source} lines
with the corresponding source files.
The program, \xref{\Tclf{tcl2cc.c}}{../Tcl/tcl2c++.c.html},
converts the OTcl code into an equivalent EmbeddedTcl object, \code{et_ns_lib}.

During initialization, invoking the method \code{EmbeddedTcl::load}
explicitly evaluates the array.
\begin{list}{---}{}
\item
  \xref{\Tclf{tcl-object.tcl}}{../Tcl/tcl-object.tcl}
  is evaluated by the method
  \fcnref{\fcn{Tcl::init}}{../Tcl/Tcl.cc}{Tcl::init};
  \fcnref{\fcn[]{Tcl\_AppInit}}{../ns-2/tclAppInit.cc}{::Tcl\_AppInit}
  invokes \fcn[]{Tcl::Init}.
  The exact command syntax for the load is:
  \begin{program}
        et_tclobject.load();
  \end{program}
\item
  Similarly,
  \xref{\nsf{tcl/lib/ns-lib.tcl}}{../ns-2/tcl/lib/ns-lib.tcl}
  is evaluated directly by \code{Tcl_AppInit} in \nsf{ns\_tclsh.cc}.
  \begin{program}
        et_ns_lib.load();
  \end{program}
\end{list}

\section{Class InstVar}
\label{sec:InstVar}

This section describes the internals of the \clsref{InstVar}{../Tcl/Tcl.cc}.
This class defines the methods and mechanisms to bind
a C++ member variable in the compiled shadow object
to a specified OTcl instance variable in the equivalent interpreted object.
The binding is set up such that the value of the variable can be
set or accessed either from within the interpreter, or from
within the compiled code at all times.

There are five instance variable classes:
\clsref{InstVarReal}{../Tcl/Tcl.cc},
\clsref{InstVarTime}{../Tcl/Tcl.cc},
\clsref{InstVarBandwidth}{../Tcl/Tcl.cc},
\clsref{InstVarInt}{../Tcl/Tcl.cc},
and \clsref{InstVarBool}{../Tcl/Tcl.cc},
corresponding to bindings for real, time, bandwidth, integer, and
boolean valued variables respectively.

We now describe the mechanism by which instance variables are set up.
We use the \clsref{InstVarReal}{../Tcl/Tcl.cc}
to illustrate the concept.
However, this mechanism is applicable to all five types of instance variables.

When setting up an interpreted variable to access a member variable,
the member functions of the class InstVar assume that they are executing
in the appropriate method execution context;
therefore, they do not query the interpreter to determine the context in
which this variable must exist.

In order to guarantee the correct method execution context,
a variable must only be bound if its class is already established within
the interpreter, and
the interpreter is currently operating on an object in that class.
Note that the former requires that when a method in a given class is
going to make its variables accessible via the interpreter,
there must be an associated 
\href{class TclClass}{Section}{sec:TclClass}
defined that identifies the appropriate class hierarchy to the interpreter.
The appropriate method execution context can therefore be created in one
of two ways.

An implicit solution occurs whenever a new TclObject is created within
the interpreter.
This sets up the method execution context within the interpreter.
When the compiled shadow object of the interpreted TclObject is created,
the constructor for that compiled object can bind its member variables
of that object
to interpreted instance variables in the context of the newly created
interpreted object.

An explicit solution is to define a \code{bind-variables} operation
within a \code{command} function, that can then be invoked via the
\code{cmd} method.
The correct method execution context is established in order to execute
the \code{cmd} method.
Likewise, the compiled code is now operating on the appropriate
shadow object, and can therefore safely bind the required member variables.

An instance variable is created by specifying the name of the
interpreted variable, and the address of the member variable in the
compiled object.
The
\fcnref{constructor}{../Tcl/Tcl.cc}{InstVar::InstVar}
for the base class InstVar 
creates an instance of the variable in the interpreter,
and then sets up a
\fcnref{trap routine}{../Tcl/Tcl.cc}{InstVar::catch_var}
to  catch all accesses to the variable through the interpreter.

Whenever the variable is read through the interpreter, the
\fcnref{trap routine}{../Tcl/Tcl.cc}{InstVar::catch_read}
is invoked just prior to the occurrence of the read.
The routine invokes the appropriate
\fcnref{\code{get} function}{../Tcl/Tcl.cc}{InstVarReal::get}
that returns the current value of the variable.
This value is then used to set the value of the interpreted variable
that is then read by the interpreter.

Likewise,
whenever the variable is set through the interpreter, the
\fcnref{trap routine}{../Tcl/Tcl.cc}{InstVar::catch_write}
is invoked just after to the write is completed.
The routine gets the current value set by the interpreter, 
and invokes the appropriate
\fcnref{\code{set} function}{../Tcl/Tcl.cc}{InstVarReal::set}
that sets the value of the compiled member to the current value set
within the interpreter.

\endinput

### Local Variables:
### mode: latex
### comment-column: 60
### backup-by-copying-when-linked: t
### file-precious-flag: nil
### End:
