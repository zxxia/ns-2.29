%\documentstyle[11pt,fullpage]{article}
%\setlength{\parindent}{0 in}
%\setlength{\parskip}{.1in}
%\setlength{\topmargin}{-0.5in}
%\setlength{\textheight}{8.5in}
%\begin{document}
\chapter{XCP: eXplicit Congestion control Protocol}
\label{chap:xcp}

XCP is a feedback-based congestion control system that uses direct,
explicit, router feedback to avoid congestion in the network.  It is
designed for both scalability and generality.  It was developed by
Dina Katabi, starting from a suggestion by Mark Handley (refer
to~\cite{Katabi02} for detailed descriptions). 
The \ns{} code for XCP which was originally developed by Dina Katabi was
modified, extended and integrated into ns-2.28 at USC/ISI. It still
continues to evolve as of today. If you are interested in looking at
Dina's original source code please go to her website at
http://www.ana.lcs.mit.edu/dina/XCP/ 

\section{What is XCP?}
\label{sec:xcp?}
XCP is a congestion control protocol that can be applied to any
transport protocol. It performs especially well in very high
delay-bandwidth product networks. Typically in large bandwidth-delay
product networks, efficiency of TCP goes down in the event of multiple of
packet losses and becomes unstable irrespective of queueing schemes
used. However in similar environments, XCP, using a control theory
based feedback
model, achieves much more efficiency, stability and fairness by
sending feedback from the network to the sender for setting the data
flow into the network.

XCP's scalability results from the fact that it requires no per-flow
state in the router to calculate the feedback.  Most router-assisted
congestion control systems maintain per-flow information used to
allocate the resources.  XCP keeps very little information in the
router, and this information is chosen to minimize both the amount of
router state and the per-packet operations needed to update that state.

For generality, XCP divides the resource allocation function between
two controllers: a congestion controller that ensures that flows use
all available capacity, and a fairness controller that ensures that
flows are allocated the capacity fairly.  Most congestion control
systems fail to make this division, much less to implement as two
conceptually distinct systems.  This division allows a clear
exposition and implementation of two basic resource allocation
functions in XCP. XCP sources send additional information about their
current round-trip times and router-assigned throughput in each
packet. XCP routers insert feedback into the packets that is
interpreted by the sources. 
  
Although XCP may be implemented for any transport protocol, however as an
initial test it has been implemented in TCP. The next section
gives details of the way XCP is implemented in \ns{}.


\section{Implementation of XCP in NS}
\label{sec:xcp in ns}

In \ns{}, the XCP implementation can be found under \nsf{xcp} directory. 
The protocol needs to be deployed in the TCP end points (source and
receiver) as well within the intermediate nodes which is mostly the
router and may sometimes be a link-layer switch as well. The end-point
part of XCP code may be found under xcp-end-sys.{cc,h} and the router
portion of the code may be found under xcp.{cc,h} and xcpq.{cc,h}. 

\subsection{Endpoints in XCP}
\label{sec:endpoints}

The end points consist of TCP source and sink agents using XCP as their
congestion control mechanism. The
intermediate node or router writes feedback in each packet header as
the delta\_throughput value, about the data capacity that may be
incremented if feedback is positive and should be decreased if
negative. When this packet reaches the receiver this delta\_throughput
value is returned to the sender in a reverse\_feedback field of a
congestion header in the returning packet, which is an ACK packet in
case of TCP. 
  
The sender upon receiving this reverse\_feedback value adjusts its
sending rate by increasing or decreasing its congestion window size as
the case maybe. 

The packet header that is used by XCP is implemented as a structure
called hdr\_xcp in \ns{} and is defined as follows:
\begin{program}
  double	throughput_;
  double	rtt_;
  enum \{
    XCP_DISABLED = 0,
    XCP_ENABLED,
    XCP_ACK,
  \} 	xcp_enabled_;		// to indicate that the flow is XCP enabled
  bool	xcp_sparse_;    // flag used with xcp_sparse extension
  int	        xcpId_;		// Sender's ID (debugging only)
  double	cwnd_;		// The current window (debugging only) 
  double	reverse_feedback_; 
  
  // --- Initialized by source and Updated by Router 
  double delta_throughput_;
  unsigned int controlling_hop_; 
\end{program}
  
    The xcp receiver is responsible for copying the delta\_throughput
    value into the reverse\_feedback field of the ack packets. In some
    cases where delayed ack is used the receiver calculates the sum of
    the delta\_throughput values in arriving packets for copying into the
    reverse\_feedback field of the outgoing ack packet.

    The controlling\_hop\_ field that carries the address of the router
    who has last updated the feedback is used for debugging purposes only.

    In case of a packet loss in the network, TCP's Van Jacobson
    congestion control should most likely override XCP.However in \ns
    this happens 
    a little differently. With receiving of duplicate acks indicating
    packet loss, the cwnd gets halved and fast retransmit and fast
    recovery algorithms come into play. However xcp routers continue to send
    feedback to the source based on which the source tries to open its
    cwnd. So it seems to be a mish-mash of VJCC and XCP
    algorithms. However for most cases this issue doesnot arise as XCP
    router very rarely experiences a pkt drop as the queue is
    continuously regulated and drained by XCP. Understanding the correct
    behaviour of XCP in face of pkt loss is an area of current study and
    shall be implemented in the future. 

    \subsection{XCP Router}
    \label {sec:xcp_wrapper}
    The XCP router consists of a wrapper class that holds virtual queues
    for XCP, TCP and OTHER traffic flows. OTHER flow maybe anything other
    than XCP and TCP. In the current implementation, the XCP queue is a
    drop-tail queue while those for TCP and OTHER use RED. 

    These underlying queues are bundled in a
    wrapper class XCPWrapQ that provides necessary interface to the XCP
    router. The XCP/TCP/OTHER queues are serviced in a Weighted Round-Robin
    manner. Each queue has a weight that determines the percentage of the
    service it receives. The current queue weights of 0.5 each for the XCP
    and TCP allows equal service between the two. The third queue reserved
    for OTHER flows has not been used as yet and has a weight of 0.0. 
    
    OTCL Class Queue/XCP has a flag named tcp\_xcp\_on\_ which is set to
    a default 
    value of 0. This should be set to 1 for those simulations that use
    both XCP and TCP flows. This flag is used to split the link capacity
    of the router between the XCP and TCP queues in simulations that
    use both flow types. This is supposed to be a temporary fix and
    should go away once the dynamic queue weights come into effect. A
    caveat for the tcp\_xcp flag is that it is set as an OTcl class variable
    and not per instance variable. This might cause some 
    problems in topologies that uses mix of intermittent xcp and tcp
    flows for which some router would require to support both TCP and
    XCP and some wouldn't.
    
    Every packet received by the wrapper queue class is first marked or
    assigned a code point depending on the type of the packet. Packets,
    for the current TCP implementation, are marked for XCP, TCP/TCP-ACK
    and OTHER packets. This code point is used to enque packets in the right
    queue. The wrapper class is implemented in xcp.{cc,h}.
    
    
    \subsection{XCP queue}
    \label{sec:xcp_queue}

    The XCP queue is responsible for sending back feedback in every packet
    header which is used by the sender to control rate of sending data
    packets into the network. XCP uses two control algorithms, the
    congestion controller and the fairness controller that are executed
    once every estimation control interval, Te. 

    In \ns{}  the
    estimation\_timer is used to maintain this interval which is based on
    the average rtt values of the (xcp) flows seen through the router. However
    there may be better ways of defining this interval. The outstanding
    queue in the router is measured at a separate interval Tq, for which a
    queue\_timer is used. Finally an rtt\_timer is used to measure certain
    parameters in the router like packet drops, queue size, utilization 
    etc for a given interval Tr, that may either be set by user from tcl
    scripts or it may use the highest rtt value seen for all flows in the
    router. 

    The rtt\_timer interval value, Tr maybe set from tcl using the
    following API: 
    
    \code{ $queue queue-sample-everyrtt $rtt_value}
    
    where \$queue is a handle to the xcp router and \$rtt\_value is the
    interval for which xcp queue parameters like packet drop , queue size etc
    shall be measured. See example script under
    \nsf{tcl/ex/xcp/parking\_lot\_topo/parking\_lot\_topo.tcl} for use of
    this API.
    
    On packet arrival the total input traffic seen at the XCP queue is
    incremented by the size of the packet received. The sum of inversed
    throughput and sum of rtt by throughput is incremented as
    well. Values for throughput and rtt are read from the xcp header as
    set by the TCP source. Each value is normalised by the packet size.
    
    On the event of the estimation timer going off, average rtt of all
    flows is estimated using the above two sums as follows

    \code { avg_rtt = sum_rtt_by_throughput/ sum_inv_throughput }
    
    The aggregate feedback is calculated based on the available bandwidth
    capacity of the router, arriving traffic bandwidth and the persistent
    queue length in the router. Further detailed explanation of
    calculations used by the XCP router algorithm can be found in XCP
    specification available from XCP's website at
    http://www.isi.edu/isi-xcp/docs/draft-falk-xcp-spec-00.txt 
    
    Each packet carries the current throughput value of the flow and a
    throughput adjustment or the delta\_throughput in its header. The XCP
    router based on the feedback values it calculates in the estimation
    control timeout, calculates a per-packet throughput adjustment
    feedback for every packet. Positive feedback is applied equally
    per-flow while negative feedback is made proportional to each flow's
    capacity. Also a downsream router can change the delta\_throughput
    value in a packet's header only if the feedback value calculated is
    less than that in the header (written by an less congested upstream
    router). The implementation of XCP queue in \ns{} may be found in
    xcpq.\{cc,h\}. 
    
  
    \section{XCP example script}
    \label{sec:example}
    
    Let's walk through a simple xcp script that is similar to
    \nsf{tcl/ex/xcp/xcp\_test.tcl} 
    The example uses a small dumbbell topology having 3 xcp sources
    running over a bottleneck link.
    
    The topology is setup using the node and link creation APIs. The bottleneck
    is a duplex link that has a xcp router in both directions. For
    details on creating nodes, links etc in \ns{} see Marc Greis' NS
    tutorial at http://www.isi.edu/nsnam/ns/tutorial.
  
    The bottleneck link having a XCP queue is created as follows:
  \begin{program}
    set R0 [$ns node]       ;# create Bottleneck between nodes R0 and R1 
    set R1 [$ns node]
    $ns duplex-link $R0 $R1 <BW>Mb <delay>ms XCP 
  \end{program} %$

  The side links connecting source nodes to the bottleneck link have
  XCP queues as well. 
  The API \code{queue-limit} allows users to set the buffer size in the queue.
  
  The xcp source and sink is created as follows (very similar to tcp):
  \begin{program}
    set xcp [new Agent/TCP/Reno/XCP]
    $ns attach-agent $src_node $xcp
    set xcp_sink [new Agent/XCPSink]
    $ns attach-agent $rcvr_node $xcp_sink
    $ns connect $xcp $xcp_sink
    ...
    ...
  \end{program} %$
  
  There is a tcl class GeneralSender used in the example script that
  sets up xcp agents in the source nodes and then connects them to the
  xcp receiver in the destination node. An FTP source is used in all the
  3 sources. 

  Note that once the topology is set up the link bandwidth information
  needs to be propagated to the xcp queue as this is used by the xcp
  router for feedback calculation. So for every xcp queue use the
  following tcl command:
  
  \code{ $xcp_queue set-link-capacity <bandwidth_in_bits_per_sec>}
  %$
  Next we need to trace variables in the xcp router and xcp
  sources. The GeneralSender class procedure trace-xcp sets up tracing
  for xcp sources using variable-tracing in \ns{}. 
  
  \begin{program}
    GeneralSender instproc trace-xcp parameters \{
      $self instvar tcp_ id_ tcpTrace_
      global ftracetcp$id_ 
      set ftracetcp$id_ [open  xcp$id_.tr  w]
      set tcpTrace_ [set ftracetcp$id_]
      $tcp_ attach-trace [set ftracetcp$id_]
      if \{ -1 < [lsearch $parameters cwnd]  \} \{ $tcp_ tracevar cwnd_ \}
      if \{ -1 < [lsearch $parameters seqno] \} \{ $tcp_ tracevar t_seqno_ \}
      \}
  \end{program} %$
    
  For tracing xcp queue it is required to attach a file descriptor to
  the xcp queue.  
  \begin{program} 
    $xcpq attach <file-descriptor> 
  \end{program} %$
    
  This is an example of how the trace at an xcp source looks like:
  \begin{program}
    0.00000  2  0  1  0  cwnd_ 1.000 
    0.00000  2  0  1  0  t_seqno_ 0
    0.079 x x x x throughput 0.1
    0.07900  2  0  1  0  t_seqno_ 1
    0.119064 x x x x reverse_feedback_ 0
    0.119064 x x x x controlling_hop_ 0
    0.119064 x x x x newcwnd 1
    0.11906  2  0  1  0  cwnd_ 2.000 
    0.119064 x x x x throughput 50000
    0.11906  2  0  1  0  t_seqno_ 2
    0.119064 x x x x throughput 50000
    0.11906  2  0  1  0  t_seqno_ 3
  \end{program} %$
  
  The first field gives the timestamp; the next 4 fields give the
  source id (node/port) and destination id (node/port) for the xcp
  flow. The next field gives the name of the variable being traced
  followed by the value of the variable. Note that variables like 
  cwnd\_, t\_seqno\_ are using variable tracing which is a function
  supported by the OTcl lib. While variables like throughput,
  reverse\_feedback use the XCPAgent class function trace\_var defined
  in xcp-end-sys.cc. For more on variable tracing in \ns{} please read
  section 3.4.3 in the ns manual at
  http://www.isi.edu/nsnam/ns/doc/index.html  
    
    
  And example of trace output at a xcp bottleneck router looks like below:
  \begin{program}
    Tq_ 0.0472859 0.025
    queue_bytes_ 0.0472859 0
    routerId_ 0.0472859 0
    pos_fbk 0.053544 0
    neg_fbk 0.053544 0
    delta_throughput 0.053544 0
    Thruput2 0.053544 60000
    pos_fbk 0.054024 0
    neg_fbk 0.054024 0
    delta_throughput 0.054024 0
    Thruput2 0.054024 60000
    residue_pos_fbk_not_allocated 0.0638023 0
    residue_neg_fbk_not_allocated 0.0638023 0
    input_traffic_bytes_ 0.0638023 2480
    avg_rtt_ 0.0638023 0.04
  \end{program}
  
  Here the first field describes the name of the variable, the
  second field gives the timestamp and the third field gives the
  value of the variable. The XCPQueue class function \code{trace_var()}
  is used to trace variables in the xcp queue.
  
  Additionally packet traces may be created in \ns{} using the following
  tcl APIs:
  \begin{program}
    set f_all [open out.tr w]
    $ns trace-all $f_all
  \end{program}
  
  First open a file and then attach the file descriptor to the \ns{}
  trace object such that a trace of each packet as it travels through
  the network is logged and dumped into the output file.
  
  An example of such a file would look like:
  \begin{program}
    + 0.003 4 0 xcp 40 ------- 2 4.0 1.2 0 0
    - 0.003 4 0 xcp 40 ------- 2 4.0 1.2 0 0
    r 0.013016 4 0 xcp 40 ------- 2 4.0 1.2 0 0
    + 0.013016 0 1 xcp 40 ------- 2 4.0 1.2 0 0
    - 0.013016 0 1 xcp 40 ------- 2 4.0 1.2 0 0
    r 0.023032 0 1 xcp 40 ------- 2 4.0 1.2 0 0
    + 0.023032 1 0 ack 40 ------- 2 1.2 4.0 0 1
    - 0.023032 1 0 ack 40 ------- 2 1.2 4.0 0 1
    r 0.033048 1 0 ack 40 ------- 2 1.2 4.0 0 1
    + 0.033048 0 4 ack 40 ------- 2 1.2 4.0 0 1
    - 0.033048 0 4 ack 40 ------- 2 1.2 4.0 0 1
    r 0.043064 0 4 ack 40 ------- 2 1.2 4.0 0 1
    + 0.043064 4 0 xcp 1200 ------- 2 4.0 1.2 1 2
    - 0.043064 4 0 xcp 1200 ------- 2 4.0 1.2 1 2
    + 0.043064 4 0 xcp 1200 ------- 2 4.0 1.2 2 3
    - 0.043544 4 0 xcp 1200 ------- 2 4.0 1.2 2 3
  \end{program}

  Lets try to read the first line:

  \code{+ 0.003 4 0 xcp 40 ------- 2 4.0 1.2 0 0}
  
  + means a packet is enqueued in the queue (in node 4) as it hopped
  between node 4 to node 0. You'll find traces showing packets enqued
  (+) and then dequed (-) at the queue, after which it is transmitted
  over the link to be received by the next node. packet
  type is xcp and it is of size 40 bytes. The xcp flow has an id of 2
  and the packet header has a source node/port id of 4.0 and dest
  node/port id of 1.2 and the unique packet id is 0.
    
  \section{Test-suites for XCP}
  \label{sec:test for xcp}
  
  The xcp test-suite uses 3 tests. The first one is similar to the one
  we discussed in the earlier section. It consists of a dumb-bell
  topology where 3 xcp flows share a bottleneck link. The second test
  has a similar topology having 3 xcp and 1 tcp flow sharing the same
  bottleneck. And finally the last test is built on Dina Katabi's
  parking-lot experiment referred in her SIGCOMM'02 paper. It is a
  downsized version of Dina's example. The test uses a 9-hop
  link string topology. It has 10 long XCP flows that flow through the
  entire length of the chain topology. Then it has 10 XCP flows that run
  through each hop, starting at (n-1)th hop and ending at nth hop and so
  on creating the intermittent flows. And finally there are long XCP
  flows in the reverse direction, 
  starting from the last (10th) node and ending in the first (1st) node.
  There is a bottleneck at the middle of the chain topology. Thus the
  third test employs a large and complex topology and shows the
  utilization, queue length and packet drop values at every link.
  

  \section{Commands at a glance}
  \label{sec:commands-xcp}
  
  Following is a list of commands used for xcp related simulation in ns.
  \begin{flushleft}

    \code{set xcp_src [new Agent/TCP/Reno/XCP]}\\
    This command creates an XCP source agent.

    \code{set xcp_dst [new Agent/XCPSink]}\\
    This command creates an XCP sink.

    \code{$ns duplex-link $R0 $R1 <BW>Mb <delay>ms XCP}\\
    %$
    This code creates a duplex link with specified bandwidth and link
    delay using an XCP router between node R0 and R1.

    \code{$xcp_queue set-link-capacity <bandwidth_in_bits_per_sec>}\\
    %$
    This command propagates the link bandwidth information to the xcp
    queue which uses it for the router feedback calculation.

    \code{ set tfile [open tfile w] } \\
    \code{ $xcp_queue attach $tfile }\\
    This Tcl command allows a file to be attached for tracing xcp queue
    parameters. 
    
    \code{$xcp_src attach-trace <file-descriptor>} %$
    \code{$xcp_src tracevar <var-to-traced>}\\
    This command allows the xcp sources to trace variables.
    
    \code{$queue queue-sample-everyrtt $rtt_value}\\ %$
    This command allows the user to set rtt interval values at which
    variables like packet\_drops and queue lengths are measured at the
    xcp queue.

    \code{Queue/XCP set tcp_xcp_on_ 1}\\
    This flag lets tcp and xcp flows to use the same xcp router. This
    flag is a temporary fix and should go away when dynamic queue
    weights come into effect.
    
  \end{flushleft}
  %\end{document}
